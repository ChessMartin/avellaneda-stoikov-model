\documentclass[12pt]{article}

% Packages
\usepackage{amsmath}    % For advanced math typesetting
\usepackage{amsfonts}   % For mathematical fonts
\usepackage{amssymb}    % For additional mathematical symbols
\usepackage{amsthm}     % For defining theorem environments
\usepackage{graphicx}   % For including figures
\usepackage{hyperref}   % For hyperlinks in the document
\usepackage[utf8]{inputenc} % For unicode input
\usepackage{authblk}    % For author and affiliation

% Theorem environments
\newtheorem{theorem}{Theorem}[section]
\newtheorem{lemma}[theorem]{Lemma}
\newtheorem{proposition}[theorem]{Proposition}
\newtheorem{corollary}[theorem]{Corollary}

\theoremstyle{definition}
\newtheorem{definition}[theorem]{Definition}
\newtheorem{example}[theorem]{Example}

\theoremstyle{remark}
\newtheorem{remark}[theorem]{Remark}

% Title, authors, and affiliations
\title{Optimizing High-Frequency Trading Strategies: An Application of the Avellaneda-Stoikov Model to Stock Order Book Data}

\author{Martin Le Formal, Benjamin Courtois}

\begin{document}

\maketitle

\begin{abstract}
\noindent
This study explores the application of the Avellaneda-Stoikov model, a seminal framework in high-frequency trading, to the empirical realm of stock order books.
Our research objective is to adapt and fit the model's constants to real-world data, thereby bridging the gap between theoretical finance and practical trading strategies.
By employing a robust dataset encompassing bid-ask spreads, order sizes, and transaction volumes, we meticulously estimate the model parameters, including market impact factors and order arrival rates.
The paper introduces an innovative calibration methodology that leverages advanced statistical techniques to ensure an accurate fit to the stock's dynamic trading environment.
Our findings reveal that the adapted Avellaneda-Stoikov model significantly enhances the prediction of optimal bid and ask strategies, offering a novel perspective on managing inventory risk and maximizing profitability in high-frequency trading.
The implications of this study extend beyond theoretical contributions, providing actionable insights for traders and a validated approach for further academic exploration in the field of quantitative finance.
\end{abstract}


\section{Introduction}
Introduce the topic of your research, its significance, and the main objectives of your paper.

\section{Preliminaries}
Discuss any preliminary concepts, definitions, or prior work necessary to understand your research.

\section{Main Results}
Present the main results of your research, including theorems, lemmas, and proofs.

\subsection{Theorem Statements and Proofs}
\begin{theorem}
State your theorem here.
\end{theorem}

\begin{proof}
Provide the proof of your theorem here.
\end{proof}

\section{Numerical Experiments}
If applicable, describe any numerical experiments or simulations that support your theoretical results.

\section{Conclusion}
Summarize your main findings, their implications, and potential directions for future research.

\section{Acknowledgments}
Acknowledge any funding, contributions, or thanks to individuals or institutions that supported your work.

\begin{thebibliography}{99}

\bibitem{AuthorYear}
Author(s). (Year). Title of the paper. \textit{Journal Name}, Volume(Issue), page numbers.

\end{thebibliography}

\end{document}
